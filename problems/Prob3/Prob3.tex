\documentclass[../../PS3.tex]{subfiles}

\begin{document}

\section{Molecules as Harmonic Oscillators}

We're given a Hamiltonian 
\begin{equation}
	\ham = \frac{1}{2m} \sum_{n = 1}^N \left(p_n^2 + {p'}_n^2\right) + \frac{K}{2} \sum_{n = 1}^{N} \abs{r_n - {r'}_n}^2 
\end{equation}

We want to first find the partition function:
\begin{align}
	Q = \idotsint\limits_{4N} \dd[3]{\{p_i\}} \dd[3]{\{{p'}_i\}} \dd[3]{\{r_i\}} \dd[3]{\{{r'}_i\}} e^{-\beta \ham}
\end{align}

This is $N$ identical integrals, one for each particle:
\begin{align}
	Q = \left(\iiiint \dd[3]{p} \dd[3]{{p'}} \dd[3]{r} \dd[3]{{r'}} e^{-\beta \left( \frac{1}{2m}  \left(p_n^2 + {p'}_n^2\right) + \frac{K}{2} \abs{r_n - {r'}_n}^2 \right)}\right) ^N
\end{align}
It separates:
\begin{align} \label{eq:prob3a:separatedA}
	Q = \left(\left(\int \dd[3]{p} e^{-\frac{\beta}{2m} p^2}\right)^2  \int \dd[3]{r_i} \dd[3]{{r'}_i} e^{-\beta \frac{K}{2} \abs{r_n - {r'}_n}^2 }\right) ^N
\end{align}

To evaluate this, let's start with the $p$ integrals, going right away to spherical coordinates:
\begin{align}
	\int \dd[3]{p} e^{-\frac{\beta}{2m} p^2} &= 4\pi \int_0^{\infty} \dd{p} p^2 e^{-\frac{\beta}{2m} p^2} \\
	&= 4\pi (-2m) \dv{}{\beta} \int_0^{\infty} \dd{p} e^{-\frac{\beta}{2m} p^2}
\end{align}
This is a Gaussian integral we can do.
\begin{align}
	\int \dd[3]{p} e^{-\frac{\beta}{2m} p^2} &= - 8 m \pi\dv{}{\beta} \sqrt{\frac{m \pi}{2 \beta}} \\
	&= 4 m \pi \sqrt{\frac{m \pi}{2 \beta^3}} 
\end{align}
Plugging back into \eqref{eq:prob3a:separatedA},
\begin{align}
	Q &= \left( \left(4 m \pi \sqrt{\frac{m \pi}{2 \beta^3}}  \right)^2  \int \dd[3]{r_i} \dd[3]{{r'}_i} e^{-\beta \frac{K}{2} \abs{r_n - {r'}_n}^2 }\right) ^N \\
	&= \left( 16 m^2 \pi^2 \frac{m \pi}{2 \beta^3}    \int \dd[3]{r_i} \dd[3]{{r'}_i} e^{-\beta \frac{K}{2} \abs{r_n - {r'}_n}^2 }\right) ^N \\	
	\label{eq:prob3a:pdone} Q &= \left( \frac{8 m^3 \pi^3 }{ \beta^3}  \int \dd[3]{r_i} \dd[3]{{r'}_i} e^{-\beta \frac{K}{2} \abs{r_n - {r'}_n}^2 }\right) ^N	
\end{align}
I believe the standard trick for the $r$ integrals is to define a $\vec{q} = \vec{r} - \vec{r'}$ and integrate over $q$:
\begin{align}
	\int \dd[3]{r_i} \dd[3]{{r'}_i} e^{-\beta \frac{K}{2} \abs{r_n - {r'}_n}^2 } &= V \int \dd[3]{q} e^{-\beta \frac{K}{2} q^2 }
\end{align}
The factor of volume roughly accounts for translation invariance (because each $q$ integral could take place with $r$ and $r'$ shifted by a constant vector, which must be accounted for). I'm not sure how valid that makes this result. If certain molecules could be displaced unrestrictedly far from the solid, I think other things would break. In any case, I'm adding a $V$. It makes the units work out too.
\begin{align}
	\int \dd[3]{r_i} \dd[3]{{r'}_i} e^{-\beta \frac{K}{2} \abs{r_n - {r'}_n}^2 } &= V \int \dd[3]{q} e^{-\beta \frac{K}{2} q^2 }
\end{align}
Spherical coordinates:
\begin{align}
	\int \dd[3]{r_i} \dd[3]{{r'}_i} e^{-\beta \frac{K}{2} \abs{r_n - {r'}_n}^2 } &= 4\pi V \int_0^{\infty} \dd{q} q^2 e^{-\beta \frac{K}{2} q^2 } \\
	&= 4\pi \left(- \frac{2}{K} \right)V \dv{}{\beta} \int_0^{\infty} \dd{q} e^{-\beta \frac{K}{2} q^2 } \\
	&= - \frac{8 \pi V}{K} \dv{}{\beta} \sqrt{\frac{\pi}{2 \beta K}} \\
	&= \frac{4 \pi V}{K} \sqrt{\frac{\pi}{2 \beta^3 K}} \\
\end{align}
Plugging this into \eqref{eq:prob3a:pdone}, we get
\begin{align}
	Q &= \left( \frac{8 m^3 \pi^3 }{ \beta^3} \frac{4 \pi V}{K} \sqrt{\frac{\pi}{2 \beta^3 K} }\right)^N	
\end{align}

We can find the Helmholtz free energy from that:
\begin{align}
	A &= -\frac{1}{\beta} \ln{Q} \\
	&= - \frac{N}{\beta} \ln(\frac{8 m^3 \pi^3 }{ \beta^3} \frac{4 \pi V}{K} \sqrt{\frac{\pi}{2 \beta^3 K} })
\end{align}

\subsection*{(b) Specific heat}
The specific heat is $C_V = - T \pdv[2]{A}{T}$. Let's write $A$ in terms of $T$:
\begin{align}
	A &= - N \kb T \ln(8 m^3 \pi^3 \kb^3 T^3 \frac{4 \pi V}{K} \sqrt{\frac{\pi \kb^3 T^3}{2 K} }) \\
	&= - N \kb T \ln(8 m^3 \pi^3 \kb^3 \frac{4 \pi V}{K} \sqrt{\frac{\pi \kb^3}{2 K}} T^3 T^{\flatfrac32}) \\
	&= - N \kb T \ln(\xi T^{\flatfrac92})
\end{align}
Differentiating twice:
\begin{align}
	\pdv{A}{T} &= - N \kb \ln(\xi T ^{\flatfrac92}) - N \kb T \frac{1}{\xi T^{\flatfrac92}} \frac92 T^{\flatfrac72} \\ 
	\pdv{A}{T} &= - N \kb \ln(\xi T ^{\flatfrac92}) - N \kb \frac{1}{\xi} \xi \frac9	2  	
\end{align}
and finally
\begin{align}
	C_V &= - T\pdv[2]{A}{T} \\
	&= - T \pdv{}{T} \left( - N \kb \ln(\xi T ^{\flatfrac92}) - N \kb \frac{1}{\xi} \xi \frac9	2  	\right) \\
	&= T \pdv{}{T} N \kb \ln(\xi T^{\flatfrac92}) \\
	&= N \kb T \frac{1}{\xi T^{\flatfrac92}} \xi \frac92 T^{\flatfrac72} \\
	C_V &= \frac92 N \kb
\end{align}

\subsection*{(c) Mean square separation} 
We want to find the ensemble average for $q^2$, as defined earlier. We already established that the position and momentum integrals separate, so they'll cancel, and we only need to do the integral over q:
\begin{align}
	\left<q^2\right> &= \frac{4\pi V \int_0^{\infty} \dd{q} q^2 q^2 e^{-\beta \frac{K}{2} q^2 }} {4\pi V \int_0^{\infty} \dd{q} q^2 e^{-\beta \frac{K}{2} q^2 } } \\
	&= \frac{\dv[2]{}{\beta} \frac{4}{K^2} \int_0^{\infty} \dd{q} e^{-\beta \frac{K}{2} q^2 }}{\dv{}{\beta} \frac{-2}{K} \int_0^{\infty} \dd{q} e^{-\beta \frac{K}{2} q^2 }} \\
	&= -\frac{2}{K} \frac{\dv[2]{}{\beta} \sqrt{\frac{\pi}{2 \beta k}}}{\dv{}{\beta} \sqrt{\frac{\pi}{2 \beta K}}} \\
	&= -\frac{2}{K} \frac{\dv[2]{}{\beta} \sqrt{\frac{1}{ \beta }}}{\dv{}{\beta} \sqrt{\frac{1}{\beta}}} \\
	&=- \frac{2}{K} \frac{- \dv{}{\beta} \frac12 \beta^{-\flatfrac32}}{-\frac12 \beta^{-\flatfrac32}} \\
	&= -\frac{2}{K} \frac{ \dv{}{\beta} \beta^{-\flatfrac32}}{ \beta^{-\flatfrac32}} \\
	&= -\frac{2}{K} \frac{- \frac{3}{2} \beta^{-\flatfrac52}}{ \beta^{-\flatfrac32}} \\
	&= \frac{3}{\beta K}
\end{align}
This gives us our mean square separation. As we'd expect, for higher temperatures, the molecules are further separated.

\end{document}