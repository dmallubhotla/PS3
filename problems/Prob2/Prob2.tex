\documentclass[../../PS3.tex]{subfiles}

% !TeX spellcheck = en_GB
\begin{document}

\section{Langevin Function}

We are for this problem interested in considering a system with $N$ magnetic moments of magnitude $\mu$, where the $i$th moment is oriented at some angle $\theta_i$ from the vertical. Our Hamiltonian has an external field $H$, but no coupling between moments:
\begin{equation}
	\ham = - H \sum_{n = 1}^{N} \mu \cos{\theta_n}
\end{equation}

I'm making a couple assumptions. We're not told that the moments are constrained to rotate within a particular plane, so I'm going to have each moment expressed with polar angle $\theta$, (which assumes the external field is in the $z$-direction). We want to begin by finding the equilibrium magnetisation $M$. 

\subsection*{(a) Finding the magnetisation}
I think we can start by trying to minimise the free energy. We can use $A = -\kb T \ln{Q}$. To find $Q$:
\begin{align}
	Q &= \idotsint\limits_{N} \dd{\Omega_1} \dotsm  \dd{\Omega_N} e^{-\beta\ham} \\
	Q &= \idotsint\limits_{N} \dd{\Omega_1} \dotsm  \dd{\Omega_N} e^{\beta H \sum_{n = 1}^{N} \mu \cos{\theta_n}} \\
	Q &= \idotsint\limits_{N} \dd{\Omega_1} \dotsm  \dd{\Omega_N} \prod_{n=1}^{N} e^{\beta H\mu \cos{\theta_n}} 
\end{align}
All the integrals are independent of one another, because of the lack of coupling. Additionally, each integral is the same. Thus,
\begin{align}
	Q &= \left( \int \dd{\Omega}e^{\beta H\mu \cos{\theta}} \right)^{N} \\
	&= \left( \int_0^{\tpi} \dd{\phi} \int_{-1}^{1} \dd(\cos{\theta}) e^{\beta H\mu \cos{\theta}} \right)^{N}
\end{align}
This becomes
\begin{align}
	Q &= \left( 2 \pi \int_{-1}^{1} \dd(\cos{\theta}) e^{\beta H\mu \cos{\theta}} \right)^{N} \\
	Q &= \left( 2 \pi \frac{1}{\beta H \mu} \left. e^{\beta H\mu \cos{\theta}} \right|_{-1}^{1} \right)^{N} \\
	Q &= \left( 2 \pi \frac{1}{\beta H \mu} \left(e^{\beta H\mu} - e^{-\beta H \mu}\right) \right)^{N} \\ 
	Q &= \left( \frac{2 \pi }{\beta H \mu}\right)^N \left(e^{\beta H\mu} - e^{-\beta H \mu} \right)^{N} \\
	&= \left( \frac{2 \pi }{\beta H \mu}\right)^N \left(2\sinh(\beta H \mu) \right)^{N}
\end{align}
To find the average magnetisation, we want to compute 
\begin{align}
	\left< M \right> &= \frac{1}{Q} \idotsint\limits_{N} \dd{\Omega_1} \dotsm  \dd{\Omega_N} \sum_{i = 1}^N \mu \cos\theta_i  e^{-\beta\ham(\cos\theta)} 
\end{align}

For each $\theta_i$, we have $N - 1$ copies of the integral we already did for $Q$ with no $\cos \theta_i$ in the integrand, and so we can write this as 
\begin{align}
	\left< M \right> &= \frac{1}{Q} \left( \frac{2 \pi }{\beta H \mu}\right)^{N-1} \left(2\sinh(\beta H \mu) \right)^{N-1} \sum_{i = 1}^N  \int \dd{\Omega_i} \mu \cos\theta_i  e^{-\beta\ham_i} 
\end{align}
There are $N$ identical integrals being summed together, and we can cancel out the terms in $Q$. Here, $\ham_i$ represents the contribution to the Hamiltonian of the $i$th atom. Every other part of the Hamiltonian has been integrated over.
\begin{align}
	\left< M \right> &= \frac{1}{\frac{2 \pi }{\beta H \mu} 2\sinh(\beta H \mu) }  N \int \dd{\Omega_i} \mu \cos\theta_i  e^{-\beta\ham_i} \\
	\frac{M}{N} &= \frac{\int \dd{\Omega_i} \mu \cos\theta_i  e^{-\beta\ham_i} }{\frac{2 \pi }{\beta H \mu} 2\sinh(\beta H \mu)}
\end{align}
We've dropped the average brackets on $M$. Let's integrate:
\begin{align}
	\frac{M}{N} &= \frac{\int \dd{\phi} \dd{\theta} \sin\theta \mu \cos\theta  e^{H \beta \mu \cos\theta}}{\frac{2 \pi }{\beta H \mu} 2\sinh(\beta H \mu)} \\
	&= 2 \pi \mu \frac{\int_{-1}^{1} \dd(\cos\theta) \cos\theta  e^{H \beta \mu \cos\theta}}{\frac{2 \pi }{\beta H \mu} 2\sinh(\beta H \mu)} \\
	&= \frac{2 \pi \mu}{H \mu} \frac{\dv{}{\beta} \int_{-1}^{1} \dd(\cos\theta)  e^{H \beta \mu \cos\theta}}{\frac{2 \pi }{\beta H \mu} 2\sinh(\beta H \mu)} \\
	&= \frac{2 \pi \mu}{H \mu} \frac{\dv{}{\beta} \frac{1}{H \beta \mu} \left(e^{H \beta \mu } - e^{-H \beta \mu} \right)}{\frac{2 \pi }{\beta H \mu} 2\sinh(\beta H \mu)} \\
	&= \frac{2 \pi \beta H \mu}{H^2 \mu} \frac{\dv{}{\beta} \frac{1}{\beta } \sinh(\beta H \mu)}{2 \pi \sinh(\beta H \mu)} \\
	&= \frac{\beta }{H} \frac{ \frac{-1}{\beta ^2} \sinh(\beta H \mu) + \frac{1}{\beta} H \mu \cosh(\beta H \mu)}{ \sinh(\beta H \mu)} \\
	&= \frac{1}{H} \frac{ H \mu \cosh(\beta H \mu) - \frac{1}{\beta} \sinh(\beta H \mu)}{\sinh(\beta H \mu)} \\
	&= \frac{1}{H} \left(H \mu \coth(\beta H \mu) - \frac{1}{\beta} \right) \\
	M &= \mu N \left( \coth(\beta H \mu) - \frac{1}{\beta \mu H} \right)
\end{align}

This is what we wanted to show.

\subsection*{(b) Finding the susceptibility}
We want to find $\chi = \pdv{M}{H}$ for low temperatures. The Taylor series for $\coth$ is $\coth(x) \approx \frac{1}{x} + \frac{x}{3}$. Plugging this in, we find

\begin{align}
	M &\approx \mu N \left( \frac{1}{\beta H \mu} + \frac{\beta H \mu}{3} - \frac{1}{\beta H \mu} \right)\\
	M &\approx \frac{\mu N \beta H \mu}{3} \\
	M &\approx \frac{\mu^2 N \beta H }{3}
\end{align}

This gives us $\chi \approx \frac{\mu^2 N \beta}{3}$.

\subsection*{(c) Finding the Curie's law coefficient)}
Rewriting $\beta$, we get 
\begin{equation}
	\chi = \frac{\mu ^2 N}{3 \kb T},
\end{equation}
which satisfies Curie's law for $C = \frac{\mu^2 N}{3\kb}$

\end{document}
